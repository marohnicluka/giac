\documentclass[a4paper,11pt]{article}
%\textwidth 11,8 cm
%\textheight 17 cm
\textheight 23 cm
\usepackage{graphicx}
\usepackage{amsmath}
\usepackage{amsfonts}
\usepackage{amssymb}
\usepackage{stmaryrd}
\usepackage{makeidx}
\usepackage{times}
\usepackage{hevea}
%\usepackage{mathptmx}
%Uncomment next line for pdflatex and use includegraphics with eps file
% for latex2html don't use the option [width=\textwidth]
% check that xfig files are exported magnif 100%
%\usepackage[francais]{babel}
\usepackage{ifpdf}
\ifpdf
 \usepackage[pdftex,colorlinks]{hyperref}
\else
 \usepackage[ps2pdf,breaklinks=true,colorlinks=true,linkcolor=red,citecolor=green]{hyperref}
 \usepackage{pst-plot}
\fi

%\def\@evenhead{\thepage\hfill{\footnotesize\textit{\leftmark}}}
%\def\@oddhead{\footnotesize{\textit{\rightmark}}\hfill\thepage}
%\usepackage{hp}
\usepackage[utf8]{inputenc}
\usepackage[T1]{fontenc}
%\usepackage[francais]{babel}
\usepackage{latexsym}

% Inline commands; use \texttt
% Displayed commands and output, use this
\newenvironment{giaccmd}
{\begin{quote}\ttfamily}
{\end{quote}}

% For graphics
\newcommand{\includeimage}[1]
{\includegraphics[width=0.75\textwidth]{#1}}

\title{Graph theory package for Giac/Xcas}
\author{Luka Marohnić}

\begin{document}
\maketitle
\tableofcontents

\section{Introduction}

This document contains an overview of the graph theory commands built in the Giac/Xcas software.

The commands are divided into the following six sections: \emph{Constructing graphs}, \emph{Modifying graphs}, \emph{Import and export}, \emph{Graph properties}, \emph{Traversing graphs} and \emph{Visualizing graphs}.

\section{Constructing graphs}

\subsection{Creating graphs from scratch : {\tt graph}, {\tt digraph}}

The command {\tt graph} accepts between one and three mandatory arguments, each of them being one of the following structural elements of the resulting graph :
\begin{itemize}
  \item the number or list of vertices (a vertex may be any atomary object, such as an integer, a symbol or a string); it must be the first argument if used,
  \item the set of edges (each edge is a list containing two vertices), a permutation, a trail of edges or a sequence of trails; it can be either the first or the second argument if used,
  \item the adjacency or weight matrix.
\end{itemize}
Additionally, some of the following options may be appended to the sequence of arguments :
\begin{itemize}
  \item {\tt directed = true} or {\tt false},
  \item {\tt weighted = true} or {\tt false},
  \item {\tt color = }an integer or a list of integers representing color(s) of the vertices,
  \item {\tt coordinates = }a list of vertex 2D or 3D coordinates.
\end{itemize}
The {\tt graph} command may also be called by passing a string, representing the name of a special graph, as its only argument. In that case the corresponding graph will be constructed and returned. The supported graphs and their names are listed below.
\begin{itemize}
  \item Clebsch graph : {\tt clebsch}
  \item Coxeter graph : {\tt coxeter}
  \item Desargues graph : {\tt desargues}
  \item Dodecahedron graph : {\tt dodecahedron}
  \item D\"urer graph : {\tt durer}
  \item Dyck graph : {\tt dyck}
  \item Grinberg graph : {\tt grinberg}
  \item Grotzsch graph : {\tt grotzsch}
  \item Harries graph : {\tt harries}
  \item Harries--Wong graph : {\tt harries-wong}
  \item Heawood graph : {\tt heawood}
  \item Herschel graph : {\tt herschel}
  \item Icosahedron graph : {\tt icosahedron}
  \item Levi graph : {\tt levi}
  \item Ljubljana graph : {\tt ljubljana}
  \item McGee graph : {\tt mcgee}
  \item M\"obius--Kantor graph : {\tt mobius-kantor}
  \item Nauru graph : {\tt nauru}
  \item Octahedron graph : {\tt octahedron}
  \item Pappus graph : {\tt pappus}
  \item Petersen graph : {\tt petersen}
  \item Robertson graph : {\tt robertson}
  \item Soccer ball graph : {\tt soccerball}
  \item Tetrahedron graph : {\tt tehtrahedron}
\end{itemize}

The {\tt digraph} command is used for creating directed graphs, although it is also possible with the {\tt graph} command by specifying the option {\tt directed=true}. Actually, calling {\tt digraph} is the same as calling {\tt graph} with that option appended to the sequence of arguments. However, creating special graphs is not supported by {\tt digraph} since they are all undirected. Edges in directed graphs are called \emph{arcs}. Edges and arcs are different structures: an edge is represented by a two-element set containing its endpoints, while an arc is represented by the ordered pairs of its endpoints.

The following series of examples demostrates the various possibilities when using {\tt graph} and {\tt digraph} commands.

\paragraph{Creating vertices.}
A graph consisting only of vertices and no edges can be created simply by providing the number of vertices or the list of vertex labels.\\
Input :
\begin{center}
  \tt graph(5)
\end{center}
Output :
\begin{center}
  \tt an undirected unweighted graph with 5 vertices and 0 edges
\end{center}
Input :
\begin{center}
  \tt graph([a,b,c])
\end{center}
Output :
\begin{center}
  \tt an undirected unweighted graph with 3 vertices and 0 edges
\end{center}

\paragraph{Creating single edges and arcs.}
Edges/arcs must be specified inside a set so that it can be distinguished from a (adjacency or weight) matrix. If only a set of edges/arcs is specified, the vertices needed to establish these will be created automatically. Note that, when constructing a directed graph, the order of the vertices in an arc matters; in undirected graphs it is not meaningful.\\
Input :
\begin{center}
  \tt graph(\%\{[a,b],[b,c],[a,c]\%\})
\end{center}
Output :
\begin{center}
  \tt an undirected unweighted graph with 3 vertices and 3 edges
\end{center}
Edge weights may also be specified.\\
Input :
\begin{center}
  \tt graph(\%\{[[a,b],2],[[b,c],2.3],[[c,a],3/2]\%\})
\end{center}
Output :
\begin{center}
  \tt an undirected weighted graph with 3 vertices and 3 edges
\end{center}
If the graph contains isolated vertices (not connected to any other vertex) or a particular order of vertices is desired, the list of vertices has to be specified first.\\
Input :
\begin{center}
  \tt graph([d,b,c,a],\%\{[a,b],[b,c],[a,c]\%\})
\end{center}
Output :
\begin{center}
  \tt an undirected unweighted graph with 4 vertices and 3 edges
\end{center}

\paragraph{Creating paths and trails.}
A directed graph can also be created from a list of $ n $ vertices and a permutation of order $ n $. The resulting graph consists of a single directed path with the vertices ordered according to the permutation.\\
Input :
\begin{center}
  \tt graph([a,b,c,d],[1,2,3,0])
\end{center}
Output :
\begin{center}
  \tt a directed unweighted graph with 4 vertices and 3 arcs
\end{center}
Alternatively, one may specify edges as a trail.\\
Input :
\begin{center}
  \tt digraph([a,b,c,d],trail(b,c,d,a))
\end{center}
Output :
\begin{center}
  \tt a directed unweighted graph with 4 vertices and 3 arcs
\end{center}
Using trails is also possible when creating undirected graphs. Also, some vertices in a trail may be repeated.\\
Input :
\begin{center}
  \tt graph([a,b,c,d],trail(b,c,d,a,c))
\end{center}
Output :
\begin{center}
  \tt an undirected unweighted graph with 4 vertices and 3 edges
\end{center}
There is also the possibility of specifying several trails in a sequence, which is useful for designing more complex graphs.\\
Input :
\begin{center}
  \tt graph(trail(1,2,3,4,2),trail(3,5,6,7,5,4))
\end{center}
Output :
\begin{center}
  \tt an undirected unweighted graph with 7 vertices and 9 edges
\end{center}

\paragraph{Specifying adjacency or weight matrix.}
A graph can be created from a single square matrix $ A=[a_{ij}]_n $ of order $ n $. If it contains only ones and zeros and has zeros on its diagonal, it is assumed to be the adjacency matrix for the desired graph. Otherwise, if an element outside the set $ \{0,1\} $ is encountered, it is assumed that the matrix of edge weights is passed as input, causing the resulting graph to be weighted accordingly. In each case, exactly $ n $ vertices will be created and $ i $-th and $ j $-th vertex will be connected iff $ a_{ij}\neq 0 $. If the matrix is symmetric, the resulting graph will be undirected, otherwise it will be directed.\\
Input :
\begin{center}
  \tt graph([[0,1,1,0],[1,0,0,1],[1,0,0,0],[0,1,0,0]])
\end{center}
Output :
\begin{center}
  \tt an undirected unweighted graph with 4 vertices and 3 edges
\end{center}
Input :
\begin{center}
  \tt graph([[0,1.0,2.3,0],[4,0,0,3.1],[0,0,0,0],[0,0,0,0]])
\end{center}
Output :
\begin{center}
  \tt a directed weighted graph with 4 vertices and 4 arcs
\end{center}

\subsection{Promoting to directed and/or weighted graphs : {\tt make\_directed}, {\tt make\_weighted}}

\subsection{Cycle graphs : {\tt cycle\_graph}}

\subsection{Path graphs : {\tt path\_graph}}

\subsection{Trail of edges : {\tt trail}}

\subsection{Complete graphs : {\tt complete\_graph}, {\tt complete\_binary\_tree}, {\tt complete\_kary\_tree}}

\subsection{Creating graph from a graphic sequence : {\tt is\_graphic\_sequence}, {\tt sequence\_graph}}

\subsection{Interval graphs : {\tt interval\_graph}}

\subsection{Star graphs : {\tt star\_graph}}

\subsection{Wheel graphs : {\tt wheel\_graph}}

\subsection{Web graphs : {\tt web\_graph}}

\subsection{Prism graphs : {\tt prism\_graph}}

\subsection{Antiprism graphs : {\tt antiprism\_graph}}

\subsection{Grid graphs : {\tt grid\_graph}, {\tt torus\_grid\_graphs}}

\subsection{Kneser graphs : {\tt kneser\_graph}, {\tt odd\_graph}}

\subsection{Sierpinski graphs : {\tt sierpinski\_graph}}

\subsection{Generalized Petersen graphs : {\tt petersen\_graph}}

\subsection{Isomorphic copy of a graph : {\tt isomorphic\_copy}, {\tt permute\_vertices}, {\tt relabel\_vertices}}

\subsection{Extracting subgraphs of a graph : {\tt subgraph}, {\tt induced\_subgraph}}

\subsection{Graph complement : {\tt graph\_complement}}

\subsection{Union of graphs : {\tt graph\_union}, {\tt disjoint\_union}}

\subsection{Joining two graphs : {\tt graph\_join}}

\subsection{Graph power : {\tt graph\_power}}

\subsection{Underlying graph : {\tt underlying\_graph}}

\subsection{Reversing arc directions in a digraph : {\tt reverse\_graph}}

\subsection{Graph product : {\tt cartesian\_product}, {\tt tensor\_product}}

\subsection{Seidel switch : {\tt seidel\_switch}}

\section{Modifying graphs}

\section{Import and export}

\section{Graph properties}

\section{Traversing graphs}

\section{Visualizing graphs}

\end{document}
